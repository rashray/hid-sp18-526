% status: 5
% chapter: TBD

\title{CMENV: A Cloudmesh Virtual Environment}

\author{Tim Whitson}
\affiliation{%
  \institution{Indiana University}
  \streetaddress{Smith Research Center}
  \city{Bloomington}
  \state{IN}
  \postcode{47408}
  \country{USA}}
\email{tdwhitso@indiana.edu}

\author{Gregor von Laszewski}
\affiliation{%
  \institution{Indiana University}
  \streetaddress{Smith Research Center}
  \city{Bloomington}
  \state{IN}
  \postcode{47408}
  \country{USA}}
\email{laszewski@gmail.com}

% The default list of authors is too long for headers}
\renewcommand{\shortauthors}{T. Whitson}

\begin{abstract}
Cloudmesh allows users to manage multiple cloud environments with a simple
command-line interface. Many modern cloud technologies also incorporate REST
APIs into their infrastructure. Therefore, Cloudmesh must adapt. In this
paper, we propose a new Cloudmesh computing environment, \textit{cmenv},
where multiple services are configured and combined into a single REST API.
\end{abstract}

\keywords{hid-sp18-526, cloudmesh, rest, flasger, swagger}

\maketitle

\section{Introduction}

The modern business/research environment for crunching large datasets is
in the cloud. In most cases, numerous cloud environments are necessary for
each project. Currently, there exist few tools for managing multiple cloud
environments. Cloudmesh Client is a management tool for multiple cloud
environments. Currently, Cloudmesh uses a customized command shell, cmd5,
to issue commands to the Cloudmesh Client.

%FIXME: cite Cloudmesh

We propose replacing Cloudmesh client and cmd5 with a Cloudmesh virtual
environment. Each environment is instantiated from the command line and
configured via a YAML file. The YAML file contains information about which
services must be rendered, such as data store and key-value store.

To interface with the virtual environment, a REST API will be created. To
conform with OpenAPI standards, the REST service will be created in
Flasgger/Swagger. The REST API reads from the configuration file to ensure
that all requirements for the environment are met.

%FIXME: cite OpenAPI, flasger

\section{Configuration}

\section{Virtual Machine Management}

\section{Public Keys}

\section{Key-Value Store}

\section{Data Services}

% cite

\bibliographystyle{ACM-Reference-Format}
\bibliography{report}

